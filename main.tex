\documentclass[11pt,b5paper,oneside]{article}
\usepackage{khustyle}

%vvvvvvvvvvvvvvvvvvvvvvvvvvvvvvvvvvvvvvvvvvvvvvvvvvvvvvvvv%
\title{\LaTeX Template to write your Ph.D thesis or Master thesis at the Kyung Hee University}

\author{Frodo Baggins}
%\advisor{Prof. Albert Einstein}
\advisor{Prof. Gandalf Grey\\and\\ Prof. Saruman White}
\committee{Gimli Dwarves}
\committee{Legolas Elves}
\committee{Aragorn Strider}
%\committee{Boromir Gondor}
%\committee{Sauron D. Lord}

\degree{Master of Science}
%\degree{Doctor of Philosophy}
\department{Department of Everything}
\date{February, 2023}

\keywords{Kyung Hee University, LaTex template, How-to}
%^^^^^^^^^^^^^^^^^^^^^^^^^^^^^^^^^^^^^^^^^^^^^^^^^^^^^^^^^%

\begin{document}
%\def\nofigure{1} % if you want to hide list of figures or don't have any figure
%\def\notable{1} % if you want to hide list of tables or don't have any table
\makefrontmatters

\begin{abstract}%
A Ph.D thesis or Master thesis can be written with \LaTeX. There can be some templates inherited from other students or even from your advisor with partially filled contents. But it is usally better to provide a well-defined package which packs up all necessary features ready for your thesis. You don't have copy and paste cover letters, dissertion, names etc burden in very long and complicated .tex files. Rather, just set proper variables and focus on your main contents - as you did to write your journal paper.
\end{abstract}

%vvvvvvvvvvvvvvvvvvvvvvvvvvvvvvvvvvvvvvvvvvvvvvvvvvvvvvvvv%
\section{Introduction}
Writing a thesis is can be painful. You have to start from preparing well-defined \LaTeX template to start from. One of the choices can be copying entire contents from your colleagues and edit .tex files. This is usually working, but it is error-proune task. Maybe you forget to change the date, name of advisor, or even your name. The \TeX system is not designed to work in that way, one can prepare a well-formed style files which focuses on the design \cite{texbook,latex:companion,latex2e,knuth:1984,lesk:1977}. What you have to do is focus on the contents.
%^^^^^^^^^^^^^^^^^^^^^^^^^^^^^^^^^^^^^^^^^^^^^^^^^^^^^^^^^%

\newpage
\bibliography{reference}

\end{document}