\documentclass[11pt]{article}
\usepackage{khustyle}

%vvvvvvvvvvvvvvvvvvvvvvvvvvvvvvvvvvvvvvvvvvvvvvvvvvvvvvvvv%
%% Some useful packages
\usepackage{siunitx}
\usepackage{graphicx}
\usepackage{color}
%^^^^^^^^^^^^^^^^^^^^^^^^^^^^^^^^^^^^^^^^^^^^^^^^^^^^^^^^^%

%vvvvvvvvvvvvvvvvvvvvvvvvvvvvvvvvvvvvvvvvvvvvvvvvvvvvvvvvv%
\title{\LaTeX Template to write your Ph.D thesis or Master thesis at the Kyung Hee University}

\author{Frodo Baggins}
\degree{Master of Science}
%\degree{Doctor of Philosophy}
\date{February, 2023}
\department{Department of Everything}

\advisor{Prof. Gandalf the White}
\committee{Gimli Dwarves}
\committee{Legolas Elves}
\committee{Aragorn Strider}
%\committee{Boromir Gondor}
%\committee{Sauron D. Lord}

\keywords{Kyung Hee University, LaTex template, How-to}
%^^^^^^^^^^^^^^^^^^^^^^^^^^^^^^^^^^^^^^^^^^^^^^^^^^^^^^^^^%

\begin{document}
%\def\nofigure{1} % if you want to hide list of figures or don't have any figure
%\def\notable{1} % if you want to hide list of tables or don't have any table
\makefrontmatters % DO NOT REMOVE

\begin{abstract}%
A Ph.D thesis or Master thesis can be written with \LaTeX. There can be some templates inherited from other students or even from your advisor with partially filled contents. But it is usally better to provide a well-defined package which packs up all necessary features ready for your thesis. You don't have copy and paste cover letters, dissertion, names etc burden in very long and complicated .tex files. Rather, just set proper variables and focus on your main contents - as you did to write your journal paper.
\end{abstract}

%vvvvvvvvvvvvvvvvvvvvvvvvvvvvvvvvvvvvvvvvvvvvvvvvvvvvvvvvv%
\section{Introduction}
Writing a thesis is can be painful. You have to start from preparing well-defined \LaTeX template to start from. One of the choices can be copying entire contents from your colleagues and edit .tex files. This is usually working, but it is error-proune task. Maybe you forget to change the date, name of advisor, or even your name. The \TeX system is not designed to work in that way, one can prepare a well-formed style files which focuses on the design \cite{texbook,latex:companion,latex2e,knuth:1984,lesk:1977}. What you have to do is focus on the contents.

\section{Features}
Except for the front-maters, there should be no special items to be added. You have to call few commands to set proper names, dates. You can find list of commands to be appeared {\textbf before the {\texttt\textbackslash{}begin\{document\}}} command in the Tab. \ref{tab:commands}.

\begin{table}[h]
    \centering
    \caption{List of commands to be called in this template}\label{tab:commands}
    \begin{tabular}{r|l}
        \hline
        Command & Description \\
        \hline\hline
        {\texttt\textbackslash{}title\{TITLE\}} & The title of this thesis \\
        {\texttt\textbackslash{}author\{AUTHOR\}} & Your name \\
        {\texttt\textbackslash{}advisor\{ADVISOR\}} & Your thesis advisor's name \\
        {\texttt\textbackslash{}committee\{COMMITTEE\}} & Name of thesis comittee(s). You can call this command multiple times \\
        {\texttt\textbackslash{}degree\{Master of XXX/Doctor of Philosophy\}} & Which degree you are aquiring? \\
        {\texttt\textbackslash{}date\{February/August, 20XX\}} & The Month/Year when you get your degree \\
        {\texttt\textbackslash{}department\{Department of XXX\}} & Your department name \\
        {\texttt\textbackslash{}keyword\{Item1, Item2\}} & Keyword of this thesis \\
        {\texttt\textbackslash{}def\textbackslash{}nofigure\{1\}} & (optional) do not show list of figures \\
        {\texttt\textbackslash{}def\textbackslash{}notable\{1\}} & (optional) do not show list of tables \\
        \hline
    \end{tabular}
\end{table}

\section{Conclusion}
Just enjoy writing. 

If you have any questions or request, please contact Junghwan Goh (jhgoh@khu.ac.kr).
%^^^^^^^^^^^^^^^^^^^^^^^^^^^^^^^^^^^^^^^^^^^^^^^^^^^^^^^^^%

\newpage
\bibliography{reference}

\end{document}