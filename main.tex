\documentclass[11pt,b5paper,oneside]{article}
\usepackage{khustyle}

%vvvvvvvvvvvvvvvvvvvvvvvvvvvvvvvvvvvvvvvvvvvvvvvvvvvvvvvvv%
\title{\LaTeX Template to write your Ph.D thesis or Master thesis at the Kyung Hee University}
\author{Frodo Baggins}
%\advisor{Prof. Albert Einstein}
\advisor{Prof. Gandalf Grey\\and\\ Prof. Saruman White}
\degree{Master of Science}
%\degree{Doctor of Philosophy}
\department{Department of Everything}
\date{February, 2023}
\keywords{Kyung Hee University, LaTex template, How-to}
%^^^^^^^^^^^^^^^^^^^^^^^^^^^^^^^^^^^^^^^^^^^^^^^^^^^^^^^^^%

\begin{document}
\committee{Gimli Dwarves}
\committee{Legolas Elves}
\committee{Aragorn Strider}
\committee{Boromir Gondor}
\committee{Sauron D. Lord}
%\def\nofigure{1} % if you want to hide list of figures or don't have any figure
%\def\notable{1} % if you want to hide list of tables or don't have any table
\makefrontmatters

\begin{abstract}%
A Ph.D thesis or Master thesis can be written with \LaTeX. There can be some templates inherited from other students or even from your advisor with partially filled contents. But it is usally better to provide a well-defined package which packs up all necessary features ready for your thesis. You don't have copy and paste cover letters, dissertion, names etc burden in very long and complicated .tex files. Rather, just set proper variables and focus on your main contents - as you did to write your journal paper.
\end{abstract}

%vvvvvvvvvvvvvvvvvvvvvvvvvvvvvvvvvvvvvvvvvvvvvvvvvvvvvvvvv%
\section{Introduction}

%^^^^^^^^^^^^^^^^^^^^^^^^^^^^^^^^^^^^^^^^^^^^^^^^^^^^^^^^^%

\end{document}